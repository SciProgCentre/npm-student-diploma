\chapter{Некоторые функции}

\section{Crystal Ball}\label{sec:crystalball}

Функция Crystal Ball --- функция плотности вероятности, часто используемая для описания потерь энергии в физике высоких энергий.
$$f(x;\alpha,n,\bar x,\sigma) = N \cdot \begin{cases} \exp(- \frac{(x - \bar x)^2}{2 \sigma^2}), & \mbox{for }\frac{x - \bar x}{\sigma} > -\alpha \\
 A \cdot (B - \frac{x - \bar x}{\sigma})^{-n}, & \mbox{for }\frac{x - \bar x}{\sigma} \leq -\alpha \end{cases}$$
где
$$A = \left(\frac{n}{\left| \alpha \right|}\right)^n \cdot \exp\left(- \frac {\left| \alpha \right|^2}{2}\right),$$
$$B = \frac{n}{\left| \alpha \right|}  - \left| \alpha \right|,$$
$$N = \frac{1}{\sigma (C + D)},$$
$$C = \frac{n}{\left| \alpha \right|} \cdot \frac{1}{n-1} \cdot \exp\left(- \frac {\left| \alpha \right|^2}{2}\right),$$
$$D = \sqrt{\frac{\pi}{2}} \left(1 + \operatorname{erf}\left(\frac{\left| \alpha \right|}{\sqrt 2}\right)\right).$$
В работе использовалась функция Crystal Ball с параметрами $\alpha=1, n=2, x_0=0, \sigma=1$.

\section{Сигмоида}\label{sec:sigmoida}

Сигмоида --- гладкая монотонная нелинейная функция, имеющая форму буквы "S". В нашем случае использовалась сигмоида вида:
$$\sigma(x) = \frac{1}{1+e^{-x}}.$$
