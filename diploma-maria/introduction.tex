\likechapter{Введение}

Наблюдение узких особенностей представляется принципиальной задачей любой (оптической, атомной, ядерной) спектроскопии. В постановке спектрометрических задач экспериментальной физики, как правило, существует противоречие между улучшением разрешаюшей способности прибора для поиска указанных особенностей и интенсивностью сигнала, определяющей статистическую значимость. В тех случаях, когда разрешение улучшить нельзя, и искомая особенность искажена, для ее ``восстановления'' возникает необходимость в решении обратной задачи.
И такие задачи, как правило, некорректны. Это значит, что малые (статистические) флуктуации в измеряемой функции, обуславливают большие флуктуации искомой функции - решения.
В качестве примера некорректно поставленной задачи в работе рассматривается уравнение Фредгольма I рода. Математическая формулировка задачи такова: по известным оператору $\hat{K}$ и функции $f$, найти функцию $\varphi$, являющуюся решением уравнения:
\begin{equation}
\hat{K}\varphi = f.
\label{eq:opereq}
\end{equation}
Данная математическая модель описывает следующую прикладную задачу: восстановить некоторый закон природы (входной сигнал измерительной аппаратуры) по проведенным в эксперименте измерениям (выходному сигналу), при условии, что известна аппаратная функция прибора (оператор, преобразующий входной сигнал в выходной). Поскольку функция $f$ соответствует измеряемым данным, она всегда известна только приближенно. Тогда для некоторых операторов $\hat{K}$, задача окажется некорректной, и малый шум в исходных данных приведет к значительным искажениям в восстановленной функции $\varphi$. 

Разработано достаточно методов решения некорректных обратных задач. Самым распространненым в экспериментальной физике считается метод регуляризации Тихонова \cite{tihonov}. Но существует также иной подход к регуляризации, разработанный Турчиным \cite{turchin}, который основан на Байесовой статистике. Он имеет ряд преимуществ, однако в течении многих лет оставался без внимания ученых. 