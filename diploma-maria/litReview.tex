\likechapter{Литературный обзор}
Статистический метод регуляризации был разработан Турчиным и описан в статье \cite{turchin}. В диссертации \cite{turovceva} рассматривается применение этого метода для решения некоторых задач атмосферной физики. Данные две работы послужили основой теории настоящей выпускной квалификационной работы.


Методы с использованием Байесовой статистики для регуляризации рассмотрены в статьях \cite{review1}, \cite{rev1995}, \cite{rev:emper}, однако без рассмотрения дополнительных возможностей, таких как определение погрешности найденного решения и добавления дополнительной априорной информации. В статье \cite{rev2009} обсуждается решение проблемы выбора наилучшего значения параметра регуляризации с использованием метода максимального правдоподобия. Также ранее уже рассматривалось использование Байесовой статистики для восстановления изображений \cite{rev1972}, обработки данных позитронно-эмиссионной томографии \cite{rev1985a} и однофотонной эмиссионной компьютерной томографии \cite{rev1985b}.