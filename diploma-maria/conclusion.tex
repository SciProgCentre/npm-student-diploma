\likechapter{Заключение}

Был реализован и исследован метод статистической регуляризации Турчина, с использованием трех различных методов алгебраизации. В результате работы сделаны следующие выводы: 
\begin{itemize}
\item Существенное преимущество метода регуляризации Турчина заключаются в существовании конкретного аналитического способа нахождения параметра регуляризации и погрешностей найденного решения.
 
\item Анализ восстановления модельных задач показал зависимость результата регуляризации от способа алгебраизации, а также параметров выбранного базиса (количество базисных функций, количество узлов). Однако для каждого базиса существуют соответствующие оптимальные параметры, следовательно для каждой задачи возможно подобрать наиболее подходящий базис.

\item Метод статистической регуляризации Турчина основан на Байесовом подходе, что дает возможность использования дополнительной априорной информации. Было реализовано добавление дополнительной априорной информации о неотрицательности функции путем оптимизации уже найденного решения, а также добавление граничных условий для базиса B-сплайнов. Использование апостериорной вероятности предполагает возможную реализацию всей доступной априорной информации о свойствах искомой функции.

\item Данный метод регуляризации был применен для обработки данных калибровочных измерений с эксперимента ``Троицк ню-масс''. Результаты регуляризации хорошо согласуются с результатами, полученными ранее при использовании стандартных методов обработки, что подтверждает работоспособность метода в условиях реального физического эксперимента.
 \end{itemize}
 
\hfill \break
\hfill \break

Автор данной работы выражает глубокую признательность своему научному руководителю Нозику Александру Аркадьевичу за постановку интересной задачи, активное руководство, а также проявленное терпение и внимание. 

Я также глубоко благодарна Михаилу Зеленому и Алексею Худякову за неоценимую помощь, многочисленные советы, обсуждения и терпеливые объяснения.

Мне приятно также поблагодарить Нозика Валерия Зиновьевича за полезные обсуждения полученных результатов и помощь в подготовке данной дипломной работы.
 
 